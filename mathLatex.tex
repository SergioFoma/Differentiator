\documentclass[12pt, litterpaper]{article}
\usepackage{graphicx}
\usepackage{mathtext}
\usepackage[utf8]{inputenc}
\usepackage[english, russian]{babel}
\usepackage{float}
\usepackage{tabularx}
\begin{document}
\textb{Дифференцирование математический функций в LaTeX:}
\newline
\newline
\[ \frac{ d }{ dx }( \frac{ 17 }{ 23 }) = \frac{ 0 \cdot  23 - 17 \cdot  0 }{ 23 ^{ 2 }} \]
 \[ \frac{ d }{ dx }(  34 \cdot  34 ) =  0 \cdot  34 + 34 \cdot  0  \]
 \[ \frac{ d }{ dx }( \left( 17 - 5 \right)) =  0 - 0  \]
 \[ \frac{ d }{ dx }( sin  \left( \left( 17 - 5 \right) \right) ) = cos  \left( \left( 17 - 5 \right) \right) \cdot \left( 0 - 0 \right) \]
 \[ \frac{ d }{ dx }( sin  \left( \left( 17 - 5 \right) \right) - 34 \cdot  34 ) = cos  \left( \left( 17 - 5 \right) \right) \cdot \left( 0 - 0 \right)- 0 \cdot  34 + 34 \cdot  0  \]
 \[ \frac{ d }{ dx }( sin  \left( \left( 17 - 5 \right) \right) - 34 \cdot  34 +\frac{ 17 }{ 23 }) = cos  \left( \left( 17 - 5 \right) \right) \cdot \left( 0 - 0 \right)- 0 \cdot  34 + 34 \cdot  0 +\frac{ 0 \cdot  23 - 17 \cdot  0 }{ 23 ^{ 2 }} \]
 
\end{document}